\section{Discussion}
\label{sec:discussion}

As the world switch is primarily triggered by SMC instruction, the
super-privileged attacker in legacy OS can just block the call of SMC
instruction to mount the deny-of-service attack to prevent the user from
switching to TrustFA Agent in secure world. A solution would be to add a special
button on the smartphone and set the corresponding interrupt as secure. Once the
button is pressed, secure kernel will be triggered immediately.

The attacker can also mount the man-in-the-middle (MIM) attack by executing a
counterfeit of TrustFA Agent in normal world. Instead of the real TrustFA
Agent, the user will interact with the counterfeit. As the counterfeit is not
able to retrieve $K_{private}$ in secure world, the photos not encrypted with
the private key cannot pass the verification of the remote server.

As mentioned in section \ref{sec:implementation}, many peripheral drivers are
going to be ported to the secure kernel, which will increase the TCB size.
TrustFA is orthogonal to TrustUI \cite{TrustUI} minimizing the TCB in secure
world by placing only the driver's wrapper in secure world.
