\section{Related Work}
\label{sec:related}

\noindent
{\bf Biometric Authentication~} 
Many biometric authentication techniques are involved in our life.
Fingerprint-based authentication is introduced with Apple's iPhone
\cite{iphone}. Karthikeyan \etal \cite{cmu-finger} compare the usability of
Apple’s iPhone 5S Touch ID fingerprint-based authentication with PIN-based
authentication and concludes that the former is better than the latter from
usability standpoint. In addition, face unlock is also implemented on Android
\cite{face-unlock}.  A fundamental limitation of biometric authentication is
that it is not very difficult for attacker to covertly obtain a person's photo
and video from social networks, or fingerprint pattern from an object or surface
touched by a person.  Researchers have developed numerous facial liveness
detection techniques \cite{Jain-MSU}, e.g., to capture spontaneous eye blinks or
lip movements \cite{Pan-ICCV}.  While it is useful for photo attacks, it cannot
deal with recorded videos.  3D face recognition has been widely studied in the
recent years \cite{Wang-PAMI, Amberg-FG, Chaua-3D}.  The 3D capturing process is
much more time consuming than 2D methods or entering a password.  Chen \etal
\cite{Chen-Sensor} make the first effort to employ motion sensors of smartphones
to improve the performance and security of facial authentication.  Our work also
utilize smartphone accelerometer to correlate 2D photo with 3D facial model and
leverage the Nose Angle Detection algorithm in \cite{Chen-Sensor}.  Regarding
Phase 2, our work assume a stronger threat model than \cite{Chen-Sensor}. While
\cite{Chen-Sensor} assumes the legacy OS is trusted, we assume it is potentially
malicious. In addition, the computation overhead of virtual camera attack
detection of \cite{Chen-Sensor} in Phase 2 is relatively high. By leveraging TrustZone, the overhead
of machine learning related computing can be eliminated and user only needs to
pay for the overhead of TrustZone world switch, which is in the magnitude of microsecond. 


\noindent
{\bf ARM TrustZone~}
TrustZone is the security extension of ARM. One of its application is to
protect the integrity of the OS kernel. TrustDump \cite{TrustDump} is a
TrustZone-based memory acquisition mechanism to reliably obtain the RAM memory
and CPU registers of the mobile OS kernel even if the OS has crashed or has
been compromised.  TZ-RKP \cite{TZ-RKP} and SPROBES \cite{SPROBES} propose
real-time OS protection mechanisms where the kernel is instrumented and all
critical kernel modifications will be trapped to TrustZone secure world.
Besides, TrustZone has been used to protect sensitive data.  DroidVault
\cite{DroidVault} establishes a secure channel between data owners and data
users while allowing data owners to enforce strong control over the sensitive
data with a minimal TCB in TrustZone secure world. TrustUI \cite{TrustUI}
proposes a new trusted path design for mobile devices that enables secure
interaction between end users and services based on ARM's TrustZone technology,
which is orthogonal to our work.  TLR \cite{TLR} enables the separation of
application security-sensitive logic from the rest of the application, and
isolates it from the OS and other apps. VeriUI \cite{VeriUI} introduces a TrustZone-assisted
credential-based authentication.  Unlike \cite{TrustUI}, our TrustFA is a
TrustZone-assisted facial authentication and we leverage the accelerometer to
different 2D counterfeit from 3D face.  Considering the urge requirement of
trusted reading from sensors such as GPS, camera, or microphones, Liu \etal
\cite{TrustSensor} implement a software abstraction for trusted sensors. In
our prototype, we will just implement our own software interfaces.
